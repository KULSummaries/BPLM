\section{Inleiding}
De focus van dit vak ligt op het inzichten verwerven uit wiskundige berekeningen aangezien het een kwantitatief vak is. De inzichten zijn dus het belangrijkste.

PLM gaat over alle \textit{transformatieprocessen} die inputs omzetten in outputs en bijgevolg toegevoegde waarde cre\"eren. Hierbij zijn drie aspecten van belang:
\begin{itemize}
    \item Tijd: Hoe lang duurt het proces, wat drijft die tijd en kunnen we het versnellen?
    \item Voorraad: hoeveel, waar en hoe lang aanhouden?
    \item Output: Hoeveel kunnen we maken en hoe moeten we investeren om meer te kunnen produceren?
\end{itemize}

Het \textit{doel} van deze transformatieprocessen is: ``matching supply with demand''. Op macro-economisch gaan is het namelijk zo dat:
\begin{itemize}
    \item als $S > D \Rightarrow p \downarrow$
    \item als $D > S \Rightarrow p \uparrow$
\end{itemize}
Op bedrijfseconomisch niveau is dit echter niet zo en zal het volgende gebeuren:
\begin{itemize}
    \item $S > D \Rightarrow$ overschot/voorraad $\Rightarrow$ geld zit tijdelijk vast in je voorraad en je kan het niet elders gebruiken = \textit{voorraad-kost}
    \item $D > S \Rightarrow$ tekorst en klant zou naar concurrentie kunnen gaan = \textit{tekort-kost}
\end{itemize}

Het punt waar $S = D$ zal in de praktijk bijna nooit voorkomen $\Rightarrow$ zoeken naar evenwicht waar kost het laagst is.

\subsection{Globalisering}
Globalisering: ``the world is flat''. Er zijn geen grenzen als het op PLM aankomt. Overal in de wereld kan je produceren en er wordt ook niet meer enkel voor de lokale markt geproduceerd. Dit zorgde voor het onstaan van \textit{footloose bedrijven}: bedrijven die niet meer gevestigd zijn in 1 locatie.

Deze Globale supply chain brengt ook veel onzekerheid met zich mee vanwege economische (vb. olieprijs), politieke (loonstijgingen in China) en andere factoren.
